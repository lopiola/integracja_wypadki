\section{Cel dokumentu}\label{cel-dokumentu}

Niniejszy dokument ma na celu przedstawienie hipotez, które stawiamy
sobie w projekcie przed preprowadzeniem analizy danych. Pod kątem tych
własnie hipotez będziemy przeprowadzać analizy a następnie opisane
zostaną wyniki i wnioski wraz z komentarzem, czy hipoteza została
potwierdzona czy obalona i z czego wynika taki a nie inny rezultat.

\section{Hipotezy}\label{hipotezy}

W ramach analizy, chcemy nie tylko przeanalizować wpływ pojedynczych
atrybutów na występowanie wypadków, ale także zweryfikować bardziej
zaawansowane hipotezy dotyczące wpływu złożonych czynników na
bezpieczeństwo na drodze.

\textbf{Numer:} 1\\\textbf{Nazwa:} Ograniczenie
widoczności\\\textbf{Treść:} Czynniki ograniczające widoczność powinny
mieć duży wpływ na wzrost liczby wypadków. W szczególności groźne są
kombinacje takich czynników, przykładowo, niezwykle groźnymi warunkami
na drodze są połączenie noc i mgła, czy noc i mgła i deszcz. Można w tę
analizę włączyć jeszcze stan oświetlenia - brak oświetlenia na drodze
może dodatkowo pogarszać warunki.

\textbf{Numer:} 2\\\textbf{Nazwa:} Ograniczenie widoczności i
piesi\\\textbf{Treść:} Warunki ograniczenia widoczności mogą powodować
większą liczbę wypadków z udziałem pieszych. Piesi są najmniej
uprzywilejowanymi uczestnikami ruchu i są też w trudnych warunkach
najmniej widoczni, szczególnie w wypadku braku odblasków.

\textbf{Numer:} 3\\\textbf{Nazwa:} Niesprzyjające warunki atmosferyczne
a ostrożność kierowców\\\textbf{Treść:} Deszcz lub śnieg albo ich
połączenie są groźnymi warunkami do jazdy. Dodatkowo silny wiatr może
sprawić, iż kierowca ma ograniczoną kontrolę nad samochodem. Należy
jednak sprawdzić, czy fakt, że w trudnych warunkach kierowcy jeżdżą
zdecydowanie ostrożniej i nie decydują się na brawurowe zachowania tak
często jak w dobrych warunkach nie sprawia, że wypadków tych nie jest
tak dużo więcej jak można by się spodziewać.

\textbf{Numer:} 4\\\textbf{Nazwa:} Złe warunki a przekraczanie
prędkości\\\textbf{Treść:} W niesprzyjających warunkach (atmosferycznych
i oświetleniowych) kierowcy będą rzadziej przekraczać prędkość niż w
warunkach sprzyjających, stąd większy procent wypadków przy dodatkowym
przekroczeniu prędkości będzie w warunkach sprzyjających.

\textbf{Numer:} 5\\\textbf{Nazwa:} Alkohol we krwi kierowcy a
czas\\\textbf{Treść:} Częściej wypadki spowodowane obecnością alkoholu
we krwi kierowcy będą zdarzać się w okolicach świąt, wieczorami i w
weekendy.

\textbf{Numer:} 6\\\textbf{Nazwa:} Przekraczanie prędkości a
czas\\\textbf{Treść:} Rzadziej wypadki spowodowane przekroczeniem
prędkości przez kierowcę będą zdarzać się w zimie i na jesień niż w
pozostałych porach roku, gdyż kierowcy są ostrożniejsi w trudniejszych
warunkach.

\textbf{Numer:} 7\\\textbf{Nazwa:} Wypadki z udziałem pieszych a
czas\\\textbf{Treść:} Wypadki z udziałem pieszych mogą być częstsze w
weekendy oraz na wiosnę i w lecie, wtedy pieszych na drogach jest
więcej.

\textbf{Numer:} 8\\\textbf{Nazwa:} Liczba wypadków a
czas\\\textbf{Treść:} W ciągu dnia wypadków może być więcej w godzinach
szczytu, wieczorem w okolicach zmroku, kiedy widoczność jest najgorsza.
W ciągu roku mogłoby być ich więcej w zimie, z powodu gorszych warunków.
W skali roku na poziomie dni może ich być njawięcej w okolicach świąt,
gdyż jest wtedy wzmożony ruch i więcej pijanych kierowców (patrz
hipoteza 5).

\textbf{Numer:} 9\\\textbf{Nazwa:} Wypadki a wiek
kierowców\\\textbf{Treść:} Najwięcej wypadków będzie powodowanych przez
kierowców młodych ze względu na brawurę i brak doświadczenia.

\textbf{Numer:} 10\\\textbf{Nazwa:} Wypadki a rodzaj
drogi\\\textbf{Treść:} Na autostradach będzie mało wypadków, ponieważ są
one bezpiecznymi drogami - obowiązuje na nich nieskomplikowany sposób
poruszania się, nie ma skrzyżowań, oraz mieszkańcy USA i WB są
przyzwyczajeni do częstego korzystania z nich. Ponadto nie poruszają się
po nich piesi. Natomiast więcej wypadków może pojawić się na drogach o
randze porównywalnej z polskimi krajowymi oraz wojewódzkimi (Principal,
Major), ponieważ z jednej strony mogą mieć skomplikowana infrastrukturę
co może wymagać więcej umiejętności od kierowców, a z drugiej pozwalają
osiągnąć znaczne prędkości. Możliwe jest też zaobserwowanie znacznej
ilości wypadków śmiertelnych na drogach drugorzędnych (Minor), ponieważ
często na terenach wiejskich ludzie dopuszczają się bardziej brawurowej
jazdy i kierowania po spożyciu, gdyż rzadziej można tam spotkać
funkcjonariuszy drogówki.
