\section{Cel dokumentu}\label{cel-dokumentu}

Niniejszy dokument ma za zadanie sprecyzowanie kreteriów analizy
przyczyn wypadków. Prezentuje dane i atrybuty, na których skupimy się w
naszej analizie, gdyż podejrzewamy, że mogą mieć wpływ na powstawanie
wypadków drogowych

\section{Kryteria analizy}\label{kryteria-analizy}

\subsection{data}\label{data}

\begin{itemize}
\itemsep-14pt\parskip0pt\parsep0pt
\item
  \textbf{pora dnia:} Należy dokonać rozróżnienia dzień/noc oraz
  bardziej szczegółowego - rano/popołudnie/\ldots{} Przewidujemy, że noc
  może być czynnikiem powodującym zwiększenie ilości wypadków z powodu
  gorszej widoczności i gorszego stanu kierowcy(senność), z drugiej
  strony mniejsza liczba samochodów na drodze obniża ryzyko kolizji.
  Okolice godzin wieczornych mogą się również okazać okresem, gdzie
  narażenie na wystąpienie wypadku jest większe - zmrok powoduje znaczne
  pogorszenie widoczności, może również wystąpić senność a ruch nadal
  jest wzmożony.\\
\item
  \textbf{pora roku:} Można wnioskować, że większa liczba wypadków
  będzie występować w zimie, z powodu cięższych warunków
  atmosferycznych, jednakże mogą one skutkować większą ostrożnością
  ukierowców.
\end{itemize}

\subsection{miejsce}\label{miejsce}

\begin{itemize}
\itemsep-14pt\parskip0pt\parsep0pt
\item
  \textbf{umiejscowienie względem skrzyżowania:} Należy zbadać, czy
  wypadki częściej występują na skrzyżowaniach czy poza nimi oraz w
  ramach wypadków na skrzyżowaniach jaka zależność występuje od rodzaju
  skrzyżowania i konkretnego miejsca na skrzyżowaniu. Skłaniamy się do
  opinii, że więcej wypadków będzie się zdarzało na skrzyżowaniach z
  powodu podwyższonego ryzyka błędu i przecinania się dróg pojazdów.\\
\item
  \textbf{umiejscowienie względem przejścia dla pieszych:} Należy
  rozważyć jak częste są wypadki w okolicach przejścia dla pieszych.
  Podejrzewamy, iż obecność przejścia dla pieszych zwiększa wyzyko
  wypadku a w szególności wypadku śmiertelnego z powodu dodatkowego
  czynnika jakim są piesi, różnicy w prędkości oraz braku
  zabezpieczeń.\\
\item
  \textbf{współrzędne GPS:} Współrzędne mogą zostać wykorzystane do
  wizualizacji danych na mapie i przykładowo wnioskowaniu o większej
  liczbie wypadków w okolicach miast.
\end{itemize}

\subsection{warunki pogodowe}\label{warunki-pogodowe}

\begin{itemize}
\itemsep-14pt\parskip0pt\parsep0pt
\item
  \textbf{opady (deszcz, śnieg):} Obecność opadów niewątpliwie wpływa na
  pogorszenie widoczności a także warunków na drodze co może się
  przyczyniać do częstszego występowania wypadków. Z drugiej strony,
  kierowcy przejawiają tendencję do wytężonej uwagi i ostrożniejszej
  jazdy w takich warunkach.\\
\item
  \textbf{mgła:} Mgła jest czynnikiem zdecydowanie pogarszającym
  widoczność i sugerującym zwiększenie ryzyka wypadku.\\
\item
  \textbf{wiatr:} Silny wiatr może powodować zwiększenie ryzyka wypadku.
\end{itemize}

\subsection{warunki środowiskowe}\label{warunki-srodowiskowe}

\begin{itemize}
\itemsep-14pt\parskip0pt\parsep0pt
\item
  \textbf{oświetlenie:} może być kluczowym czynnikiem wpływającym na
  ilość wypadków, spodziewane jest znalezienie częstych wzorców
  zawierające niekorzystne oświetlenie i pogodę jako okoliczności
  wypadków.\\
\item
  \textbf{stan nawierzchni:} posłuży do poszukiwania korelacji z danymi
  na temat poślizgu / braku przyczepności pojazdów oraz do wnioskowania
  czy stan nawierzchnii może mieć duży wpływ na ilość wypadków\\
\item
  \textbf{rodzaj (klasa) drogi:} spodziewany jest wpływ tego czynnika na
  ilość wypadków oraz przewidywane jest znalezienie korelacji z takimi
  danymi jak prędkość poruszania i przekroczenie prędkości
\end{itemize}

\subsection{dane o uczestnikach}\label{dane-o-uczestnikach}

\begin{itemize}
\itemsep-14pt\parskip0pt\parsep0pt
\item
  \textbf{wiek kierowcy:} ta informacja może posłużyć do znalezienia
  zależności pomiędzy wiekiem kierowcy a powagą wypadku, stopniem
  przekroczenia prędkości, ilością ofiar etc.\\
\item
  \textbf{poprzednie wypadki kierowcy:} w celu zweryfikowania, czy osoby
  z historią wypadków i wykroczeń powodują więcej wypadków i czy uczą
  się na błędach\\
\item
  \textbf{stan kierowcy:} informacje w rodzaju czy kierowca był zaspany
  / zmęczony / niedołężny etc. mogą posłużyć do bezpośredniego
  wnioskowania o przyczynach wypadków\\
\item
  \textbf{alkohol:} pozwoli na zobrazowanie jak dużo i jak poważnych
  wypadków powodowanych jest przez pijanych kierowców\\
\item
  \textbf{narkotyki:} informacje te pojawiają się rzadziej niż związane
  z alkoholem, ale mogą pozwolić na podobne wnioskowanie\\
\item
  \textbf{typ uczestnika (pieszy, pasażer, \ldots{}):} pozwoli uzyskać
  statyski, między innymi jak często w wypadkach biorą udział osoby
  niezmotoryzowane, ile wśród nich jest ofiar, które siedzenia w
  samochodzie są najmniej bezpieczne\\
\item
  \textbf{użycie pasów bezpieczeństwa:} czy i jak wpływa na stopień
  obrażeń\\
\item
  \textbf{otwarcie poduszek powietrznych:} czy i jak wpływa na stopień
  obrażeń\\
\item
  \textbf{użycie kasku:} czy i jak wpływa na stopień obrażeń
\end{itemize}

\subsection{dane o pojazdach}\label{dane-o-pojazdach}

\begin{itemize}
\itemsep-14pt\parskip0pt\parsep0pt
\item
  \textbf{rok produkcji:} może pozwolić na szukanie wpływu na ilość i
  powagę wypadków, ilość ofiar\\
\item
  \textbf{marka:} do celów stastycznych, możliwe jest że posiadacze
  niektórych marek (Samochody sportowe) powodują więcej wypadków\\
\item
  \textbf{model:} do celów stastycznych, możliwe jest że posiadacze
  niektórych marek (Samochody sportowe) powodują więcej wypadków\\
\item
  \textbf{typ pojazdu (np. ciężarówka):} podobnie jak wyżej, może
  posłużyć do znalezienia prawidłowości jakie rodzaje samochodów
  częściej biorą udział w wypadkach\\
\item
  \textbf{prędkość poruszania się:} istotna informacja w kontekście
  korelacji z powagą wypadku, ilością ofiar\\
\item
  \textbf{przekroczenie prędkości:} istotna informacja w kontekście
  korelacji z powagą wypadku, ilością ofiar
\end{itemize}

\section{Hipotezy do weryfikacji}\label{hipotezy-do-weryfikacji}

W ramach analizy, chcemy nie tylko przeanalizować wpływ pojedynczych
atrybutów na występowanie wypadków, ale także zweryfikować bardziej
zaawansowane hipotezy dotyczące wpływu złożonych czynników na
bezpieczeństwo na drodze.

\subsection{Ograniczenie widoczności}\label{ograniczenie-widocznosci}

Czynniki ograniczające widoczność powinny mieć duży wpływ na wzrost
liczby wypadków. W szczególności groźne są kombinacje takich czynników,
przykładowo, niezwykle groźnymi warunkami na drodze są połączenie noc i
mgła, czy noc i mgła i deszcz. Można w tę analizę włączyć jeszcze stan
oświetlenia - brak oświetlenia na drodze może dodatkowo pogarszać
warunki.

Dodatkowo można rozważyć czy warunki ograniczenia widoczności nie
powodują większej liczby wypadków z udziałem pieszych. Piesi są najmniej
uprzywilejowanymi uczestnikami ruchu i są też w trudnych warunkach
najmniej widoczni, szczególnie w wypadku braku odblasków.

\subsection{Niesprzyjające warunki
atmosferyczne}\label{niesprzyjajace-warunki-atmosferyczne}

Deszcz lub śnieg albo ich połączenie są groźnymi warunkami do jazdy.
Dodatkowo silny wiatr może sprawić, iż kierowca ma ograniczoną kontrolę
nad samochodem. Należy jednak sprawdzić, czy fakt, że w trudnych
warunkach kierowcy jeżdżą zdecydowanie ostrożniej i nie decydują się na
brawurowe zachowania tak często jak w dobrych warunkach nie sprawia, że
wypadków tych nie jest tak dużo więcej jak można by się spodziewać.

\subsection{Złe warunki a przekraczanie
prędkości}\label{ze-warunki-a-przekraczanie-predkosci}

Interesującą analizą może być sprawdzenie jak często wypadki są
powodowane w niesprzyjających warunkach (atmosferycznych i
oświetleniowych) dodatkowo z przekroczeniem prędkości przez kierowcę.
Należy to porównać z wypadkami w warunkach sprzyjających.

\subsection{Rozkład atrybutów w różnych przedziałach
czasowych}\label{rozkad-atrybutow-w-roznych-przedziaach-czasowych}

Ciekawą analizą do wykonania wydaje się rozkład czasowy wartości
niektórych atrybutów. Rozważane przedziały czasowe mogą być różne:

\begin{itemize}
\itemsep-14pt\parskip0pt\parsep0pt
\item
  pora dnia\\
\item
  dzień tygodnia\\
\item
  pora roku\\
\item
  rozkład dzienny w ciągu roku
\end{itemize}

Porównanie można przeprowadzać między innymi pod następującymi
względami:

\begin{itemize}
\itemsep-14pt\parskip0pt\parsep0pt
\item
  procent wypadków z przekroczeniem prędkości\\
\item
  procent wypadków, gdzie kierowca miał alkohol we krwi\\
\item
  średnia liczba poszkodowanych w wypadku\\
\item
  procent wypadków z udziałem pieszych\\
\item
  liczba wypadków
\end{itemize}

Analiza taka może przynieść kilka bardzo ciekawych wniosków. Przykładowo
badanie rozkładu dziennego w ciągu roku może pozwolić wskazać święta w
trakcie których zwiększa się ilość wypadków bądź wypadków pod wpływem
alkoholu. Przewidujemy, że więcej nieostrożnych i brawurowych kierowców
(alkohol i przekraczanie prędkości) może być wieczorami czy w weekendy.
Ciekawe może być też porównanie liczby wypadków pomiędzy porami roku,
gorsze warunki zimowe powinny sprawić, ale być może jest to zrównoważone
przez wzmożoną ostrożność i ograniczenie wyjeżdżania autem do minimum.
Kryzysowe mogą się okazać np.okresy przejściowe między jesienią a zimą.

\section{Zaawansowane formy
analizy}\label{zaawansowane-formy-analizy}

W ramach projektu, chcemy zastosować także ekstrakcji wiedzy z danych,
bez stawiania wcześniejszych hipotez. Cel ten możemy osiągnąć na dwa
sposoby.

Pierwszą możliwością jest przeprowadzenie klasteryzacji danych o
wypadkach. Otrzymując wyniki takiej klasteryzacji, możemy przeanalizować
podobieństwa pomiędzy wypadkami znajdującymi się w jednym klastrze i
próbować wyciągnąć cechy reprezentatywne takich wypadków i dołączyć do
analizy liczność klastrów. Można także taką analizę przeprowadzić dla
reprezentantów klastrów. Istnieje ryzyko, że taka analiza może okazać
się zbyt skomplikowana i dać jedynie ograniczoną liczbę istotnych
wniosków.

Drugim sposobem jest przeprowadzenie na danych procesu ekstrakcji
wzorców częstych. Pozwoli to na wyłonienie wartości atrybutów często
występujących wspólnie w czasie wypadków. Może byś to źródłem niezwykle
ciekawych wniosków. Należy wziąć pod uwagę, ze taka analiza jest bardzo
intensywna obliczeniowo i może być trudna do przeprowadzenia na całości
danych, należy wtedy rozważyć możliwość przeprowadzenia kilku analiz na
podzbiorze danych i połączenie pośrednich wyników analiz w wynik
całościowy.
