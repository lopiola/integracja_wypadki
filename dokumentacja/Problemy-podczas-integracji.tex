Na tej stronie zebrano opis problemów i trudności napotkanych podczas
parsowania i integrowania danych.

\begin{center}\rule{3in}{0.4pt}\end{center}

\textbf{Problem}:\\Niewielka cześć wypadków z USA nie posiada
określonego czasu wystąpienia.

\textbf{Rozwiązanie}:\\Odrzucić wypadki z brakującymi danymi, ponieważ
stanowią bardzo małą część rekordów (\textless{} 0.1\%).

\begin{center}\rule{3in}{0.4pt}\end{center}

\textbf{Problem}:\\Dane z USA publikowane są w różnych formatach w
zależności od roku wystąpienia.

\textbf{Rozwiązanie}:\\Poszukać bibliotek dla każdego używanego formatu
i użyć ich do sprowadzenia danych do jednego formatu (CSV).

\begin{center}\rule{3in}{0.4pt}\end{center}

\textbf{Problem}:\\Znaczenie wartości atrybutów w danych z USA zmienia
się w zależności od roku wystąpienia (np. wartość ``4'' w polu
``ATMOSPHERIC CONDITIONS'' znaczy ``deszcz'' w latah 1975 - 1990,
natomiast w późniejszych latach ``wiatr''). Część atrybutów przestaje
być wspierana po jakimś czasie, natomiast część pojawia się dopiero w
ostatnich latach. Ponadto, niektóry atrybuty przenoszone są pomiędzy
różnymi bazami danych.

\textbf{Rozwiązanie}:\\Zaprojektować mechanizm, który pozwoli na wygodne
mapowanie wartości atrybutów w zależności od roku wystąpienia i bazy
źródłowej. Schemat mapowania mógłby być zapisywany w pliku, co pozwoli
na łatwe parsowanie wszystkich danych po stworzeniu odpowiednich
schematów.

\begin{center}\rule{3in}{0.4pt}\end{center}

\textbf{Problem}:\\Bardzo duża ilość danych w zestawach z USA i GB.

\textbf{Rozwiązanie}:\\Umieścić w bazie część danych z możliwie
największym przekrojem jeśli chodzi o czas wystąpienia wypadku. W tym
celu można odrzucić część wypadków z każdego roku, ale w taki sposób,
aby uzyskać przykłady rekordów z różnych miesięcy.

\begin{center}\rule{3in}{0.4pt}\end{center}

\textbf{Problem}:\\Część rekordów posiada sporo nieokreślonych wartości
atrybutów.

\textbf{Rozwiązanie}:\\Ustalić limit nieokreślonych wartości, poniżej
którego dane będą odrzucane. Należy kierować się tym, aby zostawić
wypadki z wystarczającą do wnioskowania ilością informacji. Limit nie
może być zbyt wysoki, aby nie spowodować odrzucenia dużej części danych.

\begin{center}\rule{3in}{0.4pt}\end{center}

\textbf{Problem}:\\Część atrybutów posiada bardzo duży zakres możliwych
wartości, np. marka i model samochodu.

\textbf{Rozwiązanie}:\\Zastosować jakąś metodę przechowywania możliwych
wartości, np. w plikach tekstowych albo tabeli bazy danych, aby łatwo
było się do nich odnosić podczas parsowania danych.

\begin{center}\rule{3in}{0.4pt}\end{center}

\textbf{Problem}:\\Dane z Wielkiej Brytanii nie mają dokładnego wieku
osób, tylko zakwalifikowanie do przedziału wiekowego. Ponadto okazuje
się że może danych o wieku nie być w ogóle.

\textbf{Rozwiązanie}:\\Zastąpić przedział wiekowy reprezentatywną
wartością - np. średnią, albo losować wartość z przedziału wiekowego z
rozkładem jednostajnym. Tam gdzie wiek nie jest dostępny, zapisujemy do
bazy wiek ujemny, tak aby było wiadomo, iż brak tu danych.

\begin{center}\rule{3in}{0.4pt}\end{center}

\textbf{Problem}:\\Dane z Wielkiej Brytanii nie mają danych o wszystkich
pasażerach. Niektóre pojazdy posiadają według dostępnych danych 0
pasażerów.

\textbf{Rozwiązanie}:\\Ignorować wartość tego pola dla danych z Wielkiej
Brytanii. Niemożliwe jest uzupełnienie tych danych a szkoda stracić
wartość badawczą jaką ten atrybut daje dla danych z USA.
