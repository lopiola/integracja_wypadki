\subsection{Opis hipotezy}\label{opis-hipotezy}

\textbf{Numer:} 4\\\textbf{Nazwa:} Złe warunki a przekraczanie
prędkości\\\textbf{Treść:} W niesprzyjających warunkach (atmosferycznych
i oświetleniowych) kierowcy będą rzadziej przekraczać prędkość niż w
warunkach sprzyjających, stąd większy procent wypadków przy dodatkowym
przekroczeniu prędkości będzie w warunkach sprzyjających.

\subsection{Wyniki związane z
hipotezą}\label{wyniki-zwiazane-z-hipoteza}

Procent wypadków, w których przekroczono prędkość w różnych warunkach

\begin{longtable}[c]{@{}ll@{}}
\toprule\addlinespace
warunki & procent
\\\addlinespace
\midrule\endhead
wszystkie & 17.79
\\\addlinespace
deszcz & 12.60
\\\addlinespace
śnieg & 6.50
\\\addlinespace
mgła & 14.76
\\\addlinespace
ciemność & 19.83
\\\addlinespace
\bottomrule
\end{longtable}

\subsection{Weryfikacja i wnioski}\label{weryfikacja-i-wnioski}

Hipoteza znajduje zdecydowane potwierdzenie w danych. Najbardziej
drastyczny (trzykrotny) spadek udziału wypadków z przekroczeniem
prędkości obserwujemy, gdy pada śnieg. Trudne warunki na drodze
wymuszają często jazdę dużo poniżej ograniczenia i powodują zwiększenie
ostrożności u kierowców.

Ciekawy jest pzrost udziału wypadków z przekroczeniem prędkości w nocy.
Jest to prawdopodobnie spowodowane faktem, że w nocy na drogach jest
pusto. Jeżeli nie ma innych czynników ograniczających widoczność,
kierowcy czują się na drodze pewniej i jadą szybciej.
