\section{Temat projektu}\label{temat-projektu}

Tematem projektu jest integracja różnych źródeł danych opisujących
przyczyny wypadków drogowych. Temat obejmuje identyfikację
reprezentatywnych źródeł i zebranie z nich danych, sprowadzenie ich do
wspólnej reprezentacji a następnie poddanie tak przetworzonych danych
wieloprzekrojowej analizie.

\section{Cel projektu}\label{cel-projektu}

Głównym celem projektu jest analiza danych na temat wypadków drogowych
pod kątem prawdopodobnych przyczyn.

Można wyróżnić kilka pośrednich celów, których realizacja będzie
konieczna w ramach projektu. Pierwszym z nich jest zgromadzenie danych
na temat wypadków drogowych, pochodzących z reprezentatywnych źródeł.
Obszarami, na których będziemy chcieli się skupić są Polska, Europa
Zachodnia, USA i Kanada. Celem naszym jest znalezienie danych godnych
zaufania, możliwie pełnych i obszernych oraz analiza informacji, jakich
te dane nam dostarczają. Aby umożliwić dalszą pracę z danymi
pochodzącymi z różnych źródeł, konieczne będzie sprowadzenie ich do
wspólnej reprezentacji danych, pozwalającej na dalszą analizę.

Kolejnym celem stawianym w projekcie jest umożliwienie wnioskowania o
przyczynach wypadków. Dane o przyczynach mogą być dostępne w danych
bezpośrednio, w przeciwnym razie konieczna będzie identyfikacja i
selekcja kryteriów, pozwalających na przeprowadzenie wnioskowania w tym
zakresie.

Ostatecznie celem projektu będzie przeprowadzenie analiz danych na wielu
płaszczyznach z wykorzystaniem określonych wcześniej kryteriów. Pozwoli
to zarówno na identyfikację możliwych przyczyn poszczególnych wypadków
jak i wyciągnięcie wniosków co do ogólnych przyczyn wypadków i
okoliczności zwiększających ryzyko ich wystąpienia.
