\subsection{Opis hipotezy}\label{opis-hipotezy}

\textbf{Numer:} 2\\\textbf{Nazwa:} Ograniczenie widoczności i
piesi\\\textbf{Treść:} Warunki ograniczenia widoczności mogą powodować
większą liczbę wypadków z udziałem pieszych. Piesi są najmniej
uprzywilejowanymi uczestnikami ruchu i są też w trudnych warunkach
najmniej widoczni, szczególnie w wypadku braku odblasków.

\subsection{Wyniki związane z
hipotezą}\label{wyniki-zwiazane-z-hipoteza}

\begin{longtable}[c]{@{}ll@{}}
\toprule\addlinespace
warunki & procent
\\\addlinespace
\midrule\endhead
wszystkie & 16.36
\\\addlinespace
deszcz & 17.10
\\\addlinespace
śnieg & 10.37
\\\addlinespace
mgła & 13.72
\\\addlinespace
ciemność & 21.51
\\\addlinespace
\bottomrule
\end{longtable}

\subsection{Weryfikacja i wnioski}\label{weryfikacja-i-wnioski}

Można powiedzieć iż hipoteza potwierdziła się częściowo. Widzimy wyższy
procent wypadków z pieszymi w przypadku deszczu oraz ciemności.
Szczególnie ta druga pokazuje konieczność edukowania ludzi co do
konieczności noszenia na odzieży elementów odblaskowych, aby kierowca
miał szansę zauważyć pieszego na ulicy.

Ciekawa jest dużo niższa wartość dla śniegu i mgły. Może być to związane
z wzmożoną ostrożnością kierowców w takich warunkach oraz rzadszym
wychodzeniem pieszych na drogę, szczególnie kiedy pada śnieg. W deszczu
nie obserwujemy jednak tego efektu, a piesi są mniej widoczni, stąd już
wzrost w warunkach deszczowych.
