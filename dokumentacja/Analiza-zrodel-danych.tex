\subsubsection{Cel dokumentu}\label{cel-dokumentu}

Niniejszy dokument ma na celu podsumowanie informacji na temat źródeł
danych, które będziemy mogli wykorzystać w projekcie. Charakterystyka
dokumentów obejmuje zakres danych, możliwe problemy z ich otrzymaniem i
wykorzystaniem a także zawartość danych - informacje o wypadkach, jakie
można znaleźć w opisywanych danych. Dokument zawiera również wstępną
definicję kryteriów, w ramach których będziemy analizować przyczyny
wypadków drogowych, z uwzględnieniem informacji o źródłach.

\section{Przegląd źródeł danych}\label{przeglad-zrode-danych}

Głównymi obszarami zainteresowania przy szukaniu źródeł danych były:
Polska, Europa Zachodnia, USA, Kanada

\begin{enumerate}
\item
  Polska

  \begin{itemize}
  \item
    Polskie Obserwatorium Bezpieczeństwa Ruchu Drogowego (POBR) \cite{pobr}

    \begin{itemize}
    \item
      System składa się z dwóch części: hurtowni danych i portalu
      informacyjnego z interaktywną mapą do przeglądania danych z
      hurtowni.\\
    \item
      Dostęp jest ograniczony - bez rejestracji możliwy jedynie dostęp
      do wersji demo interaktywnej mapy oraz części zbiorczych
      statystyk.\\
    \item
      Studenci, dziennikarze, przedstawiciele społeczności lokalnych
      mogą uzyskać czasowy dostęp do pełnej bazy danych, po uprzednim
      pisemnym potwierdzeniu, że dane potrzebne będą do przygotowywanych
      opracowań.\\
    \item
      Z informacji na stronie wynika, że dane powinny być rzetelne i
      pełne.\\
    \item
      Wnioskując po wersji demo mapy interaktywnej, w hurtowni danych
      dostępne powinny być przynajmniej następujące informacje na temat
      wypadków:

      \begin{itemize}
      \item
        miejsce\\
      \item
        czas\\
      \item
        droga (numer, rodzaj)\\
      \item
        obszar zabudowany/niezabudowany\\
      \item
        informacje o uczestnikach i ofiarach wypadku

        \begin{itemize}
        \itemsep-14pt\parskip0pt\parsep0pt
        \item
          wiek\\
        \item
          płeć\\
        \item
          pojazd\\
        \item
          obrażenia\\
        \item
          ``ciężkość'' wypadku (śmiertelny, cięzki, lekki)\\
        \end{itemize}
      \item
        obecność alkoholu u sprawcy\\
      \item
        nadmierna prędkość\\
      \end{itemize}
    \end{itemize}
  \end{itemize}
\item
  Europa Zachodnia

  \begin{itemize}
  \item
    CARE \cite{care}

    \begin{itemize}
    \itemsep-14pt\parskip0pt\parsep0pt      
    \item
      Zawiera dane dostarczane przez odpowiednie instytucje ze
      wszystkich krajów Unii Europejskiej w postaci przetworzonej do
      wspólnego formatu\\
    \item
      Dostęp bezpośrednio do bazy niestety nie jest możliwy, dla
      wiadomości publicznej dostępne są tylko wybrane zbiorcze
      statystyki\\
    \end{itemize}
  \item
    Wielka Brytania \cite{wb}

    \begin{itemize}
    \item
      Dane z wypadków na drogach publicznych, zgłoszonych na policję i
      odnotowanych przy wykorzystaniu formularza STATS19.\\
    \item
      Dane dostępne dla lat 1979 - 2013\\
    \item
      Dane obejmują m.in.:

      \begin{itemize}
      \itemsep-14pt\parskip0pt\parsep0pt
      \item
        okoliczności wypadku\\
      \item
        typy pojazdów (marka i model)\\
      \item
        dane o ofiarach\\
      \item
        informacje o poziomie alkoholu w wydychanym powietrzu\\
      \end{itemize}
    \end{itemize}
  \item
    Belgia-Flandria \cite{belgia}

    \begin{itemize}
    \itemsep-14pt\parskip0pt\parsep0pt
    \item
      Dane zebrane przez National Institute of Statistics (NIS)\\
    \item
      Obejmują lata 1991 - 2013 i tylko jeden region. Wydaje się, że
      jest to zbyt mocne ograniczenie.
    \end{itemize}
  \end{itemize}
\item
  USA i Kanada

  \begin{itemize}
  \item
    NHTSA: National Automotive Sampling System (NASS) - Crashworthiness Data System (CDS) \cite{nass}

    \begin{itemize}
    \item
      Losowe próbki danych na temat wypadków o różnych skutkach (od
      małych po śmiertelne). Około 5'000 przekrojowo wybranych wypadków
      rocznie jest dokładnie badanych i dane są publikowane. Dane są
      zanonimizowane.\\
    \item
      Dostępne atrybuty (m. in.):

      \begin{itemize}
      \itemsep-14pt\parskip0pt\parsep0pt
      \item
        powaga wypadku (obrażenia, ilość rannych)\\
      \item
        pijani kierowcy\\
      \item
        data\\
      \item
        udział narkotyków\\
      \item
        rodzaj wypadku\\
      \item
        dane nt. pojazdów, wyposażenie\\
      \item
        brak danych o pogodzie, miejscu zdarzenia\\
      \end{itemize}
    \end{itemize}
  \item
    NHTSA: Fatality Analysis Reporting System (FARS) \cite{fars}

    \begin{itemize}
    \item
      \url{ftp://ftp.nhtsa.dot.gov/FARS/}\\
    \item
      Wypadki, w wyniku których odnotowano przynajmniej jedną ofiarę
      śmiertelną (max. w ciągu 30 dni po wypadku).\\
    \item
      Od roku 1975 zebrano dane z wypadków w trzech typach dokumentów:
      Accident, Vehicle, Person.\\
    \item
      Od roku 2010 dodano wiele nowych dokuemntów: Peperwork, Cevent,
      Vevent, Vsoe, Distract, Factor, Drimpair, Nmimpair, Maneuver,
      Nmprior, Nmcrash, Safetyeq, Violatn, Vision, Damage, Vindecode
      (2013).\\
    \item
      Dostępne atrybuty (m. in.):

      \begin{itemize}
      \itemsep-14pt\parskip0pt\parsep0pt
      \item
        szczegółowe dane o biorących udział w wypadku\\
      \item
        szczegółowe dane o pojazdach\\
      \item
        miejsce\\
      \item
        data\\
      \item
        rodzaj wypadku\\
      \item
        sposób dojścia do wypadku\\
      \item
        szczegóły drogi (rodzaj, odl. od skrzyżowania etc)\\
      \item
        oświetlenie\\
      \item
        pogoda\\
      \item
        pijani kierowcy\\
      \end{itemize}
    \end{itemize}
  \item
    Kanada - nie znaleziono publicznie dostępnych zestawów danych
  \end{itemize}
\end{enumerate}

\section{Dane z Wielkiej Brytanii}\label{dane-z-wielkiej-brytanii}

\textbf{Ogólne informacje}

Zbiór danych z Wielkiej Brytanii to dane policyjne z wypadków na drogach
publicznych, zgłoszonych i odnotowanych w formularzu (\cite{formularz}) który jest wypełniany przez funkcjonariuszy przy zgłoszeniu
zdarzenia. Dane są dostępne pod adresem \cite{wb}.
Geograficznie dane obejmują Anglię, Walię i Szkocję zaś czasowo obejmują
okres 1979 - 2013r.

\textbf{Format danych}

Dane są dostępne w dwóch paczkach, osobno dane dla lat 1979 - 2004 i
2005 - 2013, w postaci plików csv. Każda paczka danych zawiera trzy
pliki csv zawierające komplet danych opisujących wypadek:

\begin{itemize}
\itemsep-14pt\parskip0pt\parsep0pt
\item
  \emph{Accident} - dane ogólne na temat okoliczności wypadku\\
\item
  \emph{Casualty} - dane na temat ofiar wypadku i ich obrażeń, połączone
  logicznie z plikiem accident poprzez atrybut ACC\_Index, który
  jednoznacznie identyfikuje wypadek. Dla jednego wypadku możemy mieć
  kilka wpisów w pliku casualty, jeżeli była więcej niż jedna osoba
  poszkodowana.\\
\item
  \emph{Vehicle} - dane na temat pojazdów, które brały udział w kolizji
  i ich obrażeń, połączone logicznie z plikiem accident poprzez atrybut
  ACC\_Index. Dla jednego wypadku możemy mieć kilka wpisów w pliku
  vehicle.\\Pierwsza linijka każdego z plików zawiera nazwy atrybutów.
\end{itemize}

Nazwy atrybutów oraz ich wartości są przeniesione bezpośrednio z
formularza (\cite{formularz}). Pozwala to na interpretację kodów wartości zawartych w plikach
i na ich translację na wartości zrozumiałe dla człowieka. Dodatkową
pomoc w interpretacji wartości może stanowić dokument
STATS20 (\cite{stats20}), który zawiera dokładne informacje co do tego, jak wypełniać
formularz STATS19. Przedstawiony powyżej podział na trzy pliki
odzwierciedla podział formularza na analogiczne trzy części.

Z dodatkowych informacji istotnych dla realizacji projektu należy
zaznaczyć, że dane dotyczące części atrybutów mogą być niedostępne (lub
nie dotyczyć danego wypadku), wartość takiego atrybutu jest wtedy równa
-1. Zgodnie z decyzją projektową ograniczenia się do wypadków
śmiertelnych jest istotna również możliwość przefiltrowania danych i
wybrania wyłącznie wypadków śmiertelnych. Jest to możliwe dzięki polu
Casualty\_Severity (ciężkość wypadku) w pliku dotyczącym ofiar wypadku,
które może przyjmować jedną z trzech wartości: fatal, serious i slight
(śmiertelny, poważny, lekki).

\textbf{Ilość danych}

Paczka danych z lat 1979 - 2004 zawiera następującą ilość danych:

\begin{itemize}
\itemsep-14pt\parskip0pt\parsep0pt
\item
  6224198 wypadków\\
\item
  8264687 ofiar\\
\item
  10981968 pojazdów\\Paczka danych z lat 2005 - 2013 zawiera następującą
  ilość danych:\\
\item
  1494275 wypadków\\
\item
  2022243 ofiar\\
\item
  2735898 pojazdów
\end{itemize}

Pozostaje do ustalenia jaki procent tych danych stanowią dane o
wypadkach śmiertelnych

\textbf{Atrybuty}

\textbf{\emph{Szczegóły dot. wypadku}}

\begin{itemize}
\itemsep-14pt\parskip0pt\parsep0pt
\item
  data i godzina\\
\item
  dzień tygodnia\\
\item
  miejsce wypadku (długość i szerokość geograficzna lub współrzędne
  OSGR)\\
\item
  powaga wypadku\\
\item
  lekki\\
\item
  poważny\\
\item
  śmiertelny\\
\item
  szczegóły dróg (pierwszej i drugiej)\\
\item
  klasa drogi\\
\item
  numer drogi\\
\item
  liczba ofiar\\
\item
  liczba pojazdów\\
\item
  typ drogi\\
\item
  rondo\\
\item
  droga jednokierunkowa\\
\item
  jezdnia podwójna\\
\item
  jezdnia pojedyncza\\
\item
  droga dojazdowa\\
\item
  nieznany\\
\item
  ograniczenie prędkości na drodze\\
\item
  szczegóły dot. skrzyżowania (np. dalej niż 20 metrów od skrzyżowania,
  na rondzie, na wyjeździe z drogi prywatnej)\\
\item
  sposób kierowania ruchem (dla wypadków na skrzyżowaniu)\\
\item
  szczegóły dot. przejścia dla pieszych i kontroli nad nim (np. dalej
  niż 50m od najbliższego przejścia, czy na przejściu kontrola osób
  autoryzowanych)\\
\item
  rodzaj przejścia dla pieszych\\
\item
  pogoda\\
\item
  dobra, bez porywistego wiatru\\
\item
  deszcz, bez porywistego wiatru\\
\item
  śnieg, bez porywistego wiatru\\
\item
  dobra, porywisty wiatr\\
\item
  deszcz, porywisty wiatr\\
\item
  śnieg, porywisty wiatr\\
\item
  mgła\\
\item
  inne\\
\item
  nieznane\\
\item
  stan nawierzchni\\
\item
  sucha\\
\item
  mokra/wilgotna\\
\item
  śnieg\\
\item
  mróz/lód\\
\item
  zalana (powyżej 3cm wody)\\
\item
  światło\\
\item
  światło dzienne\\
\item
  ciemność, oświetlenie, zapalone\\
\item
  ciemność, oświetlenie, nie zapalone\\
\item
  ciemność, brak oświetlenia\\
\item
  ciemność, brak danych co do oświetlenia\\
\item
  warunki nadzwyczajne (np. niedziałające światła, roboty na drodze,
  uszkodzona nawierzchnia)\\
\item
  zagrożenia na jezdni (np. obiekty na jezdni, udział w poprzedzającym
  wypadku, pieszy na jezdni, zwierzę na jezdni)\\
\item
  Obecność policjanta na miejscu wypadku\\
\item
  Obszar zabudowany/niezabudowany
\end{itemize}

\textbf{\emph{Szczegóły dot. pojazdu i kierowcy}}

\begin{itemize}
\itemsep-14pt\parskip0pt\parsep0pt
\item
  kierownica po lewej stronie\\
\item
  typ samochodu (podział na 10 kategorii)\\
\item
  pojemność silnika\\
\item
  kod napędu (propulsion code)\\
\item
  pojazdy z naczepami i przegubowe\\
\item
  wiek kierowcy\\
\item
  kod pocztowy kierowcy\\
\item
  ucieczka z miejsca zdarzenia (hit and run)\\
\item
  test alkoholu w wydychanym powietrzu u kierowcy (?)\\
\item
  nie dotyczy\\
\item
  pozytywny\\
\item
  nie proszony o test\\
\item
  nie zgodził się na test\\
\item
  nie było kontaktu z kierowcą w momencie wypadku\\
\item
  nie podano (powody zdrowotne)\\
\item
  płeć kierowcy\\
\item
  umiejscowienie pojazdu w momencie zderzenia (np. na głównej drodze, na
  pasie dla tramwajów, autobusów lub rowerów)\\
\item
  umiejscowienie pojazdu na skrzyżowaniu (np. w odległości ponad 20m,
  wjeżdżający na skrzyżowanie, zjeżdżający, wjeżdżający na rondo,
  zjeżdżający z głównej drogi lub wjeżdżający na nią, na środku
  skrzyżowania)\\
\item
  wykonywany manewr (np. cofanie, zaparkowany, zatrzymany, zatrzymywanie
  się, skręcanie, zawracanie)\\
\item
  obiekt uderzony na jezdni (np. poprzedni wypadek, zaparkowany pojazd,
  most, otwarte drzwi pojazdu, krawężnik)\\
\item
  miejsce zjazdu z jezdni\\
\item
  poślizg i dachowanie\\
\item
  brak poślizgu, dachowania, jack-knifingu (złożenie się samochodu z
  naczepą na kształt scyzoryka)\\
\item
  poślizg\\
\item
  poślizg i dachowanie\\
\item
  jack-knifing\\
\item
  jack-knifing i dachowanie\\
\item
  dachowanie\\
\item
  pierwszy obiekt uderzony poza jezdnią (np. znak drogowy, latarnia,
  drzewo)\\
\item
  pierwsze miejsce uderzenia\\
\item
  przód\\
\item
  tył\\
\item
  lewy bok\\
\item
  prawy bok\\
\item
  powód podróży\\
\item
  w ramach pracy\\
\item
  do/z pracy\\
\item
  wiezienie dziecka do/ze szkoły\\
\item
  inne\\
\item
  nieznane\\
\item
  kierunek jazdy pojazdu, 10 możliwości, włączając samochód zaparkowany
\end{itemize}

\textbf{\emph{Szczegóły dot. poszkodowanych}}

\begin{itemize}
\itemsep-14pt\parskip0pt\parsep0pt
\item
  w którym pojeździe znajdowała się ofiara\\
\item
  kod pocztowy\\
\item
  płeć\\
\item
  typ poszkodowanego\\
\item
  kierowca\\
\item
  pasażer\\
\item
  pieszy\\
\item
  wiek poszkodowanego\\
\item
  powaga obrażeń\\
\item
  lekkie\\
\item
  poważne\\
\item
  śmiertelne\\
\item
  umiejscowienie pieszego (np. na jezdni, na pasach, na krawężniku, na
  wysepce centralnej)\\
\item
  kierunek podążania pieszego, podobnie jak dla pojazdów 10 możliwości,
  łącznie z nieruchomym pieszym\\
\item
  ruch pieszego względem pojazdu\\
\item
  czy pieszy był pracownikiem utrzymania dróg (road maintenance
  worker)\\
\item
  czy kierujący rowerem miał na sobie kask\\
\item
  na którym siedzeniu znajdował się pasażer\\
\item
  przednie\\
\item
  tylnie\\
\item
  szczegóły pasażera autokaru lub autobusu\\
\item
  poszkodowany nie był pasażerem autobusu\\
\item
  pasażer wsiadał\\
\item
  pasażer wysiadał\\
\item
  pasażer stał\\
\item
  pasażer siedział\\
\item
  pasy bezpieczeństwa\\
\item
  nie dotyczy\\
\item
  założone, potwierdzone niezależnie\\
\item
  założone , nie potwierdzone niezależnie\\
\item
  nie założone\\
\item
  brak danych
\end{itemize}

\section{Dane z USA}\label{dane-z-usa}

\textbf{Ogólne informacje}

Zbiór danych pochodzi z portalu organizacji NHTSA (National Highway
Traffic Safety Administration). Jest to jeden z kilku publicznie
dostępnych zbiorów, nazywa się FARS (Fatality Analisys Reporting
System). Więcej informacji dostępne jest pod linkiem \cite{fars}.
Zestaw zawiera dane ze wszystkich wypadków śmiertelnych zanotowanych na
terenie Stanów Zjednoczonych. Zbiory publikowane są corocznie, w obecnej
chwili dostępne są paczki z lat 1975 - 2013.

\textbf{Format danych}

Dane dostępne są w postaci plików bazy danych w jednym z wybranych
formatów: \textbf{.sas7bdat} lub \textbf{.dbf}. Są to standaryzowane
formaty i istnieje wiele narzędzi umożliwiających ich konwersję. Każda
paczka zawiera następujące pliki:

\begin{itemize}
\itemsep-14pt\parskip0pt\parsep0pt
\item
  ACCIDENT (od 1975) - dane ogólne na temat okoliczności wypadku.\\
\item
  VEHICLE (od 1975) - informacje na temat pojazdów biorących udział w
  wypadku, mogą być skojarzone z rekordem ACCIDENT za pomocą pola
  ST\_CASE.\\
\item
  PERSON (od 1975) - informacje na temat osób (zmotoryzowanych i
  pieszych) biorących udział w wypadku. Mogą być zkojarzone z rekordami
  ACCIDENT i VEHICLE.\\
\item
  PAPERWORK (od 2010) - informacje na temat zaparkowanych pojazdów lub
  maszyn robót drogowych (biorących udział w wyapdku).\\
\item
  CEVENT (od 2010) - lista wydarzeń, które doprowadziły do wypadku.\\
\item
  VEVENT (od 2010) - opisuje sekwencje wydarzeń z CEVENT\\
\item
  VSOE (od 2010) - uproszczona baza VEVENT\\
\item
  DAMAGE (od 2010) - lista uszkodzeń pojazdów\\
\item
  DISTRACT (od 2010) - czynniki odwracające uwgę kierowców\\
\item
  DRIMPAIR (od 2010) - dane na temat niepełnosprawności kierowców\\
\item
  FACTOR (od 2010) - okoliczności pojazdów, które mogły doprowadzić do
  wypadku\\
\item
  MANEUVER (od 2010) - manewry wykonane przez kierowcę aby uniknąć
  wypadku\\
\item
  VIOLATN (od 2010) - wykroczenia kierowców\\
\item
  VISION (od 2010) - czynniki, które mogły zmniejszać widoczność\\
\item
  NMCRASH (od 2010) - nieodpowiednie zachowania osób nieporuszających
  się pojazdami\\
\item
  NMIMPAIR (od 2010) - dane nt. niepełnosprawości osób nieporuszających
  się pojazdami\\
\item
  NMPRIOR (od 2010) - dane nt. czynności wykonywanych przez osoby
  nieporuszające się pojazdami przed wypadkiem\\
\item
  SAFETYEQ (od 2010) - dane nt. wyposażenia BHP u osób nieporuszających
  się pojazdami przed wypadkiem\\
\item
  VINDECODE (od 2013) - Kody VIN pojazdów biorących udział w wypadku
\end{itemize}

\textbf{Atrybuty}

Z racji, że większość plików dostępnych jest dopiero od 2010, nie będą
one brane pod uwagę przy integracji danych. Bazy ACCIDENT, VEHICLE i
PERSON zawierają bardzo szczegółowe dane, które powinny wystarczyć do
analiz pod kątem przyczyn wypadków drogowych.

Poniżej przedstawiono jakie dane o wypadkach są dostępne w rozważanych
bazach:

\emph{\textbf{ACCIDENT}}

\begin{itemize}
\item
  liczba osób

  \begin{itemize}
  \itemsep-14pt\parskip0pt\parsep0pt
  \item
    zmotoryzowane\\
  \item
    niezmotoryzowane\\
  \end{itemize}
\item
  liczba pojazdów

  \begin{itemize}
  \itemsep-14pt\parskip0pt\parsep0pt
  \item
    poruszające się\\
  \item
    zaparkowane\\
  \item
    pracujące\\
  \end{itemize}
\item
  miejsce

  \begin{itemize}
  \itemsep-14pt\parskip0pt\parsep0pt
  \item
    stan\\
  \item
    miasto\\
  \item
    okręg\\
  \item
    rodzaj drogi\\
  \item
    numer drogi i kamień milowy\\
  \item
    wsp. GPS\\
  \item
    jurysdykcja drogi\\
  \end{itemize}
\item
  data (dzień, miesiąc, rok, godzina, minuta)\\
\item
  okoliczności

  \begin{itemize}
  \itemsep-14pt\parskip0pt\parsep0pt
  \item
    pierwsze szkodliwe wydarzenie\\
  \item
    rodzaj kolizji\\
  \item
    umiejscowienie względem skrzyżowania\\
  \item
    udział autobusu szkolnego\\
  \item
    wypadek przy torach kolejowych\\
  \item
    czas zgłoszenia\\
  \item
    czas przyjazdu służb na miejsce\\
  \item
    czas dotarcia do szpitala\\
  \item
    \textbf{przyczyny wypadku} (np dziurawa droga, ostry zakręt, warunki
    pogodowe, śliska nawierzchnia)\\
  \item
    pijani kierowcy\\
  \item
    ofiary śmiertelne\\
  \end{itemize}
\item
  oświetlenie

  \begin{itemize}
  \itemsep-14pt\parskip0pt\parsep0pt
  \item
    dzienne\\
  \item
    ciemno / ciemno, nie oświetlone\\
  \item
    ciemno ale oświetlone\\
  \item
    zmierzch\\
  \item
    świt\\
  \item
    ciemno - nieznane oświetlenie\\
  \item
    inne\\
  \item
    nieraportowane\\
  \item
    nieznane\\
  \end{itemize}
\item
  warunki pogodowe

  \begin{itemize}
  \itemsep-14pt\parskip0pt\parsep0pt
  \item
    brak niekorzystnych warunków\\
  \item
    deszcz lub mżawka\\
  \item
    deszcz ze śniegiem lub grad\\
  \item
    śnieg lub zamieć\\
  \item
    mgła, dym lub smog\\
  \item
    porywisty wiatr\\
  \item
    wiatr unoszący piach, ziemię lub pył\\
  \item
    inne\\
  \item
    zachmurzenie\\
  \item
    nie raportowane\\
  \item
    nieznane
  \end{itemize}
\end{itemize}

\emph{\textbf{VEHICLE}}

\begin{itemize}
\itemsep-14pt\parskip0pt\parsep0pt
\item
  ilość pasażerów\\
\item
  rodzaj pojazdu\\
\item
  ucieczka z miejsca wypadku\\
\item
  stan w którym pojazd został zarejestrowany\\
\item
  posiadacz samochodu\\
\item
  marka\\
\item
  model\\
\item
  typo nadwozia\\
\item
  rok produkcji\\
\item
  VIN\\
\item
  ciągnięte przyczepy\\
\item
  jackknifing\\
\item
  przedział wagowy\\
\item
  konfiguracja pojazdu (ciężarowe)\\
\item
  rodzaj naczepy\\
\item
  przewóz niebezpiecznych materiałów\\
\item
  specjalne przeznaczenie pojazdu\\
\item
  prędkość poruszania\\
\item
  dachowanie / obrót pojazdu\\
\item
  miejsce uderzenia\\
\item
  rozmiar uszkodzeń\\
\item
  usunięcie pojazdu z miejsca wypadku\\
\item
  najbardziej szkodliwe wydarzenie\\
\item
  czynniki zw. ze stanem pojazu, które mogły być \textbf{przyczyną
  wypadku}\\
\item
  pojawienie się pożaru\\
\item
  ilość ofiar\\
\item
  czy kierowca pił\\
\item
  obecność kierowcy\\
\item
  stan, gdzie wydano prawo jazdy\\
\item
  kod pocztowy kierowcy\\
\item
  wzrost kierowcy\\
\item
  waga kierowcy\\
\item
  poprzednie wypadki kierowcy (liczba)\\
\item
  poprzednie zawieszenia prawa jazdy i skazania\\
\item
  ilość mandatów za prędkość i innych\\
\item
  daty pierwszych i ostatnich skazań / zawieszeń\\
\item
  określenie stopnia prezkroczenia prędkości\\
\item
  stan kierowcy (zaspany, depresja, chory, blackout, leki, narkotyki) -
  bardzo dużo możliwych wartości\\
\item
  ograniczenie prędkości\\
\item
  nawierzchnia i jej stan\\
\item
  znaki, jakie napotkał pojazd przez wypadkiem\\
\item
  sposób poruszania przed wypadkiem\\
\item
  krytyczne wydarzenie przed wypadkiem\\
\item
  manewr wymijający\\
\item
  stabilność pojazdu przed wypadkiem (np. trzyma się drogi, poślizg)\\
\item
  pozycja na drodze przed wypadkiem\\
\item
  rodzaj wypadku
\end{itemize}

\emph{\textbf{PERSON}}

\begin{itemize}
\itemsep-14pt\parskip0pt\parsep0pt
\item
  wiek\\
\item
  płeć\\
\item
  rodzaj (pieszy, zmotoryzowany, rowerzysta etc)\\
\item
  poziom obrażeń\\
\item
  zajmowane miejsce w pojeździe\\
\item
  użycie pasów / hełmu\\
\item
  czy powyższe były używane w odpowiedni sposób\\
\item
  czy poduszka się otworzyła\\
\item
  czy ciało wyleciało z pojazdu, i którędy\\
\item
  czy osoba musiała być wydobyta przy użyciu sprzętu lub siły\\
\item
  spożycie alkoholu\\
\item
  sposób określenia spożycia\\
\item
  czy test na nietrzeźwość był wykonany\\
\item
  udział narkotyków\\
\item
  czy była transportowana do szpitala\\
\item
  śmierć na miejscu lub w drodze do szpitala\\
\item
  data śmierci\\
\item
  \textbf{przyczyny wypadku} (dla osób niebędących kierowcą)\\
\item
  czas między wypadkiem a śmiercią\\
\item
  rasa (pochodzenie)\\
\item
  pozycja przed wypadkiem (niezmotoryzowani)
\end{itemize}
